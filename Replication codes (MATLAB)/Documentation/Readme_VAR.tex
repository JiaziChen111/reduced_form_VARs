\documentclass[11pt]{article}

\usepackage{amssymb}
\usepackage{graphicx}
\usepackage{amsmath}
\usepackage{amsfonts}

\pagestyle{myheadings} \markright{F. Schorfheide: Macroeconometrics, June 2019}
\parindent0pt
\parskip1ex plus1.5ex minus0.2ex
\setcounter{secnumdepth}{5}
\setcounter{tocdepth}{2}

\voffset0cm
\topmargin0cm

%Paper Style
%twoside
\oddsidemargin1.35cm
\evensidemargin1.35cm
\textheight8.6in
\textwidth5.5in
\renewcommand{\baselinestretch}{1.2}

%Draft STYLE
%\oddsidemargin-0.5cm
%\evensidemargin0cm
%\textheight8.6in
%\textwidth5.5in
%\renewcommand{\baselinestretch}{1.5}


\newcommand{\bc}{\begin{center}}
	\newcommand{\ec}{\end{center}}
\newcommand{\be}{\begin{equation}}
\newcommand{\ee}{\end{equation}}
\newcommand{\bea}{\begin{eqnarray}}
\newcommand{\eea}{\end{eqnarray}}
\newcommand{\bean}{\begin{eqnarray*}}
	\newcommand{\eean}{\end{eqnarray*}}
\newcommand{\refline}{\raisebox{1ex}{\underline{\hspace{2cm}} \quad}}


\newtheorem{theorem}{Theorem}
\newtheorem{definition}{Definition}
\newtheorem{lemma}{Lemma}
\newtheorem{corollary}{Corollary}
\newtheorem{proposition}{Proposition}
\newtheorem{assumption}{Assumption}

\newcounter{pkt}
\newenvironment{tlist}%
{\begin{list}{(\roman{pkt})}{\usecounter{pkt}\parskip0ex\parsep0ex\itemsep0ex\topsep0ex}}%
	{\end{list}}


\def\EE{\mathord{I\kern-.35em E}}
\def\PP{\mathord{I\kern-.3em P}}
\def\QQ{\mathord{Q\kern-5pt\hbox{\raise1.1pt\hbox{\vrule height5pt}}\kern5pt}}
\def\RR{\mathord{I\kern-.3em R}}

%\bibliographystyle{chicago}

\begin{document}
	\thispagestyle{empty}
	
	{\bf Bayesian Macroeconometrics} \hfill {\bf Gerzensee Ph.D. Course, June 2019}
	
	\vspace*{0.5cm}

\subsection*{Estimating a VAR with Minnesota Prior in MATLAB}

The code for the VAR estimation is based on Del Negro and Schorfheide (2011), ``Bayesian Macroeconometrics,'' in: {\em Oxford Handbook of Bayesian Economics.}

The environment is that of a 4-variables VAR(4) given by:
$$
\mathbf{y_{t}} = \Phi_{c} + \Phi(L)\mathbf{y_{t}} + \mathbf{u_{t}} \quad \mathbf{u_{t}} \sim N(0,\mathbf{\Sigma}) \quad 
$$

The VAR contains, respectively, the log of de-trended real GDP per capita, the inflation rate (GDP deflator), a measure of nominal interest rate and real money balances. Data are quarterly and refer to the US economy for the period 1959(1):2006(4). The model is estimated using Bayesian methods. In particular, given a prior $p(\Phi,\Sigma)$ and the likelihood function $L(\mathbf{Y_{1,T}}|\mathbf{Y_{-p+1,0}},\Phi,\Sigma)$, we get a posterior density through Bayes rule:
$$
p(\Phi,\Sigma|\mathbf{Y}) \propto  L(\mathbf{Y_{1,T}}|\mathbf{Y_{-p+1,0}},\Phi,\Sigma)p(\Phi,\Sigma)
$$

We specify that $p(\Phi,\Sigma)$ follows the Inverted Wishart Multivariate Normal density, with moments derived using dummy observations from the Minnesota prior. This prior is conjugate for the likelihood function in this model. 

The file \texttt{VAR.m} estimates the above model using a direct sampling algorithm. The file returns a figure plotting the recursive averages and the posterior marginal density of the largest eigenvalues of the companion form matrix. The file also returns the marginal data density in the scalar \texttt{mdd}. 

\texttt{VAR.m} uses a number of subroutines in the folders \texttt{Minnesota Prior} and \texttt{Data} to perform this task. In particular:


\begin{itemize}
	\item \texttt{vm-loaddata.m} $\Rightarrow$ Loads the data used in the estimation;
	\item \texttt{vm-spec.m} $\Rightarrow$ It is a file that controls the hyper-parameters of the Minnesota prior;
	\item \texttt{vm-dummy.m} $\Rightarrow$ Generates dummy observations using the Minnesota prior;  
	\item \texttt{vm-mdd.m} $\Rightarrow$ Computes the marginal data density.
\end{itemize}

 


\end{document}
 